\documentclass[a4paper,10pt]{article}
\usepackage[utf8x]{inputenc}
\usepackage{listings}
\usepackage{a4wide}
\usepackage{tabularx}
\usepackage{booktabs}

\lstloadlanguages{Bash}
\lstset{basicstyle=\small\ttfamily, escapeinside={(*@}{@*)},
showstringspaces=false, breaklines=true, breakatwhitespace=true,
tabsize=4, language=Bash, numberfirstline=true,
commentstyle=\itshape\color{CommentGreen},
stringstyle=\color{DataTypeBlue}}
\newcommand{\ilcode}[1]{\lstinline|#1|}

\title{rtmdds User's Guide}
\author{Geoffrey Biggs}

\begin{document}
\maketitle

\section{Introduction}
\label{sec:intro}

rtmdds adds support for the Data Distribution Service middleware
standard to OpenRTM-aist. It provides two new types of ports for
OpenRTM-aist RT-Components: a publisher port and a subscriber port.

This software is developed at the National Institute of Advanced Industrial
Science and Technology. Approval number H23PRO-1274. This software is licensed
under the Lesser General Public License. See COPYING and COPYING.LESSER in the
source.

\section{Requirements}
\label{sec:requirements}

rtmdss requires the C++ version of OpenRTM-aist-1.0.0.

rtmdds uses the CMake build system\footnote{http://www.cmake.org/}. You
will need at least version 2.8 to be able to build the library.

rtmdds has currently only been tested with the DDS implementation
provided by RTI\footnote{http://www.rti.com}. You will need to install
RTI DDS before compiling rtmdss. As RTI DDS is a commercial product, you
must obtain a valid license from RTI. If you are at a research institute
or a university, you may be eligible for a free R\&D license. See
http://www.rti.com/downloads/dds-research.html for details.

\section{Installation}
\label{sec:installation}

\subsection{Binary}

Users of Windows can install the library using the binary installer.
This is the recommended method of installation in Windows.

\begin{enumerate}
  \item Install DDS.
  \item Set up your environment for using DDS as described in your
  implementation's documentation.
  \begin{itemize}
    \item For RTI DDS, this means adding the ndds/scripts directory to
    your PATH environment variable (so that rtiddsgen can be found) and
    setting the NDDSHOME environment variable to the location of the
    ``ndds'' directory.
  \end{itemize}
  \item Download the rtmdds installer from the website.
  \item Double-click the executable file to begin installation.
  \item Follow the instructions to install the library.
  \item You may need to restart your computer for environment variable
  changes to take effect before using the component.
\end{enumerate}

\subsection{From source}

Follow these steps to install :

\begin{enumerate}
  \item Install DDS.

  \item Set up your environment for using DDS as described in your
  implementation's documentation.
  \begin{itemize}
    \item For RTI DDS, this means adding the ndds/scripts directory to
    your PATH environment variable (so that rtiddsgen can be found) and
    setting the NDDSHOME environment variable to the location of the
    ``ndds'' directory.
  \end{itemize}

  \item Download the source, either from the repository or a source archive,
  and extract it somewhere.

  \ilcode{tar -xvzf rtmdds.tar.gz}
  \item Change to the directory containing the extracted source.

  \ilcode{cd rtmdds}
  \item Create a directory called ``build'':

  \ilcode{mkdir build}
  \item Change to that directory.

  \ilcode{cd build}
  \item Run cmake or cmake-gui.

  \ilcode{cmake ../}
  \item If no errors occurred, run make.

  \ilcode{make}
  \item Finally, install the library. Ensure the necessary permissions
  to install into the chosen prefix are available.

  \ilcode{make install}
  \item The install destination can be changed by executing ccmake and
  changing the variable \ilcode{CMAKE_INSTALL_PREFIX}.

  \ilcode{ccmake ../}
\end{enumerate}

The library is now ready for use. See the next section for instructions on
using the library.

\section{Using the library}
\label{sec:usage}

To use the library in your own components, you must create publisher and
subscriber port instances. The steps involved are:

\begin{itemize}
  \item Create an instance of RTC::DDSPubPort or RTC::DDSSubPort. The
  template parameters must be the data type to be transported by the
  port and either the DataWriter or the DataReader for that data type. A
  DataWriter must be used with a DDSPubPort and a DataReader with a
  DDSSubPort.
  \item In the constructor of your component, initialise the port
  instance with its name and the name of your data type (as a string).
  \item In the onInitialize() method, register the port with your
  component using either the addDDSPubPort() or addDDSSubPort() method.
  \item Register your data type with DDS. The way this is done will vary
  depending on your DDS implementation. The example components provided
  with rtmdds use the RTI DDS API.
  \item In the onExecute() method:
  \begin{itemize}
    \item For publishing ports, call write(), passing in a reference to
    an instance of your data type.
    \item For subscribing ports, call read(), passing in a pointer to a
    data type.
  \end{itemize}
  \item The port instance will be cleaned up automatically by the
  component when it destructs.
\end{itemize}

See the included example components for more detail.

\section{Connecting ports}
\label{sec:connecting}

The DDS ports are not DataPorts. This means that their method of
connection is different to DataPorts and ServicePorts. There are two
ways to connect a DDS port: dynamic topic and pre-configured topic.

When a DDS port is connected, it will look in the connector profile for
the topic to connect to. This is specified in the \ilcode{ddsport.topic}
property of the connector profile.

\subsection{Pre-configured topic}

If the \ilcode{ddsport.topic} property is found in the connector profile
when the DDS port is connected, its value will be used as the name of
the topic to connect the port to. This is particularly useful for
connecting a port to an existing topic. For example, the \ilcode{rtcon}
command from rtshell can be used to connect a port to an existing topic:

\ilcode{$ rtcon pubcomp0.rtc:PublisherPort -p ddsport.topic=my_topic}

Multiple ports can, of course, be connected to the same topic at the
same time:

\ilcode{$ rtcon pubcomp0.rtc:PublisherPort subcomp0.rtc:SubscriberPort -p ddsport.topic=a_topic}

A pre-configured topic can also be set in the configuration properties
of the port. If the \ilcode{ddsport.topic} property is not set in the
connector profile, then the \ilcode{default_topic} property of the port
will be checked next. This can be set in the component's configuration
file.

\subsection{Dynamic topic}

If there is not a topic already in existance that you wish to connect
your ports to, you can have one created for you dynamically. If
both the \ilcode{ddsport.topic} connector profile property and the
\ilcode{default_topic} port property are empty, a new topic will be
created by the first port in the connection. This topic will be named
after the port's name.

\ilcode{$ rtcon pubcomp0.rtc:PublisherPort subcomp0.rtc:SubscriberPort}

\section{Properties}
\label{sec:props}

The properties that can be set for a port are given in
Table~\ref{tab:props}.

\begin{table}[t]
  \centering
  \begin{tabularx}{\columnwidth}{lllX}
    \toprule
    Property & Type & Default & Effect \\
    \midrule
    verbose & Boolean & NO & Enables or disables debugging output from DDS.  \\
    qos\_file & String & & Specifies the URLs of XML files giving QoS properties for the port. See~\ref{sec:qos}. \\
    ignore\_user\_profile & Boolean & NO & Ignore the user QoS XML file. See the DDS documentation for details. \\
    ignore\_env\_profile & Boolean & NO & Ignore the environment QoS XML file. See the DDS documentation for details. \\
    ignore\_resource\_profile & Boolean & NO & Ignore the resource QoS XML file. See the DDS documentation for details. \\
    domain & Integer & 0 & Specifies the domain to operate in. Ports on different domains are invisible to each other. \\
    default\_topic & String & & Sets the default topic to connect to if the \ilcode{ddsport.topic} property is not set in the connector profile. \\
    \bottomrule
  \end{tabularx}
  \caption{Properties available on DDS ports.}
  \label{tab:props}
\end{table}

\section{QoS}
\label{sec:qos}

DDS supports an extensive array of Quality of Service properties. It is
capable of enforcing these properties in the transport at run-time,
allowing you to configure how reliable a connection between two ports is
without changing the code. These QoS properties are specified in an XML
file that is loaded by the DDS implementation. You can specify an XML
file for a port to load by setting the \ilcode{qos_file} property on that
port. This must be a set of URLs. Only one file will be loaded. The URLs
provide alternative locations for the XML file. If it cannot be found at
the first URL, the second will be tried, and so on. Specify the URLs in
the following format:

\verb+[URL1|URL2|URL3|...|URLn]+

\section{Examples}
\label{sec:examples}

A pair of example components, one publisher and one subscriber, are
provided. They are installed to \ilcode{${prefix}/share/rtmdds/examples}.
Follow these steps to build them (assuming rtmdds was installed in
\ilcode{/usr/local}):

\begin{enumerate}
  \item \ilcode{cd $HOME}
  \item \ilcode{mkdir rtmdds_examples}
  \item \ilcode{cd rtmdds_examples}
  \item \ilcode{cmake /usr/local/share/rtmdds/examples}
  \item \ilcode{make}
\end{enumerate}

The example can be executed in conjunction with rtshell for connecting
the ports.

\begin{enumerate}
  \item \ilcode{rtm-naming}
  \item \ilcode{./rtmdds_pubcomp_standalone}
  \item \ilcode{./rtmdds_subcomp_standalone}
  \item \ilcode{rtcon /localhost/pubcomp0.rtc:PublisherPort
  /localhost/subcomp0.rtc:SubscriberPort}
  \item \ilcode{rtact /localhost/pubcomp0.rtc}
  \item \ilcode{rtact /localhost/subcomp0.rtc}
\end{enumerate}

You should see data being published by \ilcode{pubcomp0.rtc} appearing in
the terminal running \ilcode{subcomp0.rtc}.

% \section{Changelog}


\end{document}
